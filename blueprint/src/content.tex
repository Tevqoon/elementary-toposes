% In this file you should put the actual content of the blueprint.
% It will be used both by the web and the print version.
% It should *not* include the \begin{document}
%
% If you want to split the blueprint content into several files then
% the current file can be a simple sequence of \input. Otherwise It
% can start with a \section or \chapter for instance.

\section{Introduction}

An \emph{elementary topos} is a Cartesian closed category with finite limits and a subobject classifier. The goal of this project is to formalise this definition in a suitable way. Namely, in mathlib Cartesian closed categories are defined via \emph{chosen} finite products, which realise finite products as an algebraic operation in a category; for each finite product diagram we \emph{choose} a specific object to serve as the product, rather than only asking for the existence of such a product. Classically, we can still, using the axiom of choice, get chosen limits from existence.

The first part of the project will be extending this to define finite chosen limits in general -- this will allow us to work more naturally in a topos, and keep with the existing definition of Cartesian closedness by getting an instance for chosen finite products from limits.

The second part will be defining a subobject classifier, which we will again see as a chosen subobject classifier -- an object of truth values $\Omega$ in the category with a monomorphism $\operatorname{true} : 1 \rightarrow \Omega$ and a unique characteristic morphism $\chi_{U} : X \rightarrow \Omega$ for any monomorphism $U \rightarrow X$.

If time permits, we will also define a chosen natural numbers object, allowing us to also consider NNO-topoi.

After the definition, the goal will be to apply it and give instances for some common examples, like finite sets, $\operatorname{Type}$, presheaves, and sheaves.

\section{Chosen Finite Limits}

The first step is to generalise chosen finite products to general finite limit cones.

\begin{definition}[Chosen Limit]
\label{def:chosen_limit}
\lean{ChosenLimit}
Let $F : J \to C$ be a functor. A \emph{chosen limit} for $F$ is a specified limit cone for $F$.
\end{definition}

\section{Subobject Classifiers}
\section{Elementary Toposes}
\section{Natural Numbers Objects (Optional)}
\section{Examples}
\subsection{Finite Sets}
\subsection{The Category of Types}
\subsection{Presheaves}
\subsection{Sheaves}
\section{Further Developments}
