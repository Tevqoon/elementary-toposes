% In this file you should put the actual content of the blueprint.
% It will be used both by the web and the print version.
% It should *not* include the \begin{document}
%   
%   If you want to split the blueprint content into several files then
%   the current file can be a simple sequence of \input. Otherwise It
%   can start with a \section or \chapter for instance.

\section{Introduction}

An \emph{elementary topos} is a Cartesian closed category with finite limits and a subobject classifier. The goal of this project is to formalise this definition in a suitable way. Namely, in mathlib Cartesian closed categories are defined via \emph{chosen} finite products, which realise finite products as an algebraic operation in a category; for each finite product diagram we \emph{choose} a specific object to serve as the product, rather than only asking for the existence of such a product. Classically, we can still, using the axiom of choice, get chosen limits from mere existence.

The first part of the project will be extending this to define finite chosen limits in general -- this will allow us to work more naturally in a topos, and keep with the existing definition of Cartesian closedness by getting an instance for chosen finite products from limits.

The second part will be defining a subobject classifier, which we will again see as a chosen subobject classifier -- an object of truth values $\Omega$ in the category with a monomorphism $\operatorname{true} : 1 \rightarrow \Omega$ and a unique characteristic morphism $\chi_{U} : X \rightarrow \Omega$ for any monomorphism $U \rightarrow X$.

If time permits, we will also define a chosen natural numbers object, allowing us to also consider NNO-topoi.

After the definition, the goal will be to apply it and give instances for some common examples, like finite sets, $\operatorname{Type}$, presheaves, and sheaves.

\section{Chosen Finite Limits}

The first step is to generalise chosen finite products to general finite limit cones. Note that chosen limits are also limits in the sense of there existing a limit cone. Being a chosen limit is a strictly stronger property.

\begin{definition}[Chosen Limit]
  \label{def:chosen-limit}
  \lean{CategoryTheory.ChosenLimit}
  \leanok
  Let $F : J \to C$ be a functor. A \emph{chosen limit} for $F$ is a specified limit cone for $F$.
\end{definition}

\begin{proposition}[Chosen Limit implies Limit]
  \label{prop:chosen-limit-implies-limit}
  \lean{CategoryTheory.hasLimit_From_chosenLimit}
  \leanok
  \uses{def:chosen-limit}
  If we have a chosen limit for a functor $F$, then $F$ has a limit.
\end{proposition}

\begin{proof}
  \leanok
  An inhabited set is nonempty.
\end{proof}

Under choice, the reverse also holds. 

\begin{proposition}[Existence of limits implies chosen limit]
  \label{prop:limit-implies-chosen-limit}
  \lean{CategoryTheory.chosenLimit_From_hasLimit}
  \leanok
\end{proposition}

\begin{proof}
  \leanok
  Classically, a nonempty set is inhabited.
\end{proof}

\begin{definition}[Chosen Limits of Shape]
  \label{def:chosen-limits-of-shape}
  \lean{CategoryTheory.ChosenLimitsOfShape}
  \uses{def:chosen-limit}
  \leanok
  A category $C$ has \emph{chosen limits of shape} $J$ if every functor $F : J \to C$ has a chosen limit.
\end{definition}

\begin{definition}[Chosen Finite Limits]
  \label{def:chosen-finite-limits}
  \lean{CategoryTheory.ChosenFiniteLimits}
  \uses{def:chosen-limits-of-shape}
  \leanok
  A category $C$ has \emph{chosen finite limits} if it has chosen limits of shape $J$ for every finite category $J$, i.~e.~ for every finite limit cone.
\end{definition}

Note that Mathlib already provides a concept of chosen finite product it uses internally for Cartesian closed categories.

\begin{definition}[Chosen Finite Products]
  \label{def:chosen-finite-products}
  \lean{CategoryTheory.ChosenFiniteProducts}
  \mathlibok
  A category $C$ has \emph{chosen finite products} if it is equipped with:
  \begin{itemize}
  \item A choice of a limit binary fan (product) for any two objects of the category.
  \item A choice of a terminal object.
  \end{itemize}
  More precisely, this consists of chosen limit cones for the pair functor $X \leftarrow \cdot \rightarrow Y$ and for the empty functor from the empty category.
\end{definition}

Note that our definition is compatible, so that a category with chosen finite limits in particular has an instance of chosen finite products.

\begin{lemma}[Chosen Finite Products from Limits]
  \label{lem:chosen-finite-products-from-limits}
  \lean{CategoryTheory.ChosenFiniteProducts_From_ChosenFiniteLimits}
  \leanok
  \uses{def:chosen-finite-limits, def:chosen-finite-products}
  If a category $C$ has chosen finite limits, then it has chosen finite products.
\end{lemma}

\begin{proof}
  We construct the limit cones for the pair and the terminal object.
  \leanok
\end{proof}

\section{Cartesian Closedness}

The notion of a Cartesian closed category is already defined in Mathlib, so we just state the definitions.

\begin{definition}[Monoidal category]
  \label{def:monoidal-category}
  \lean{CategoryTheory.MonoidalCategory}
  \mathlibok

In a \emph{monoidal category} \(\mathcal{C}\), we can take the \emph{tensor product} of objects, \(X ⊗ Y\) and of morphisms \(f ⊗ g\).
The tensor product does not need to be strictly associative on objects,
but there is a specified \emph{associator}, \(α_{X Y Z} : (X ⊗ Y) ⊗ Z ≅ X ⊗ (Y ⊗ Z)\).
There is a tensor unit \(\mathbf{1}_{\mathcal{C}}\),
with specified left and right \emph{unitor isomorphisms}
\(λ_X : \mathbf{1}_{\mathcal{C}} ⊗ X ≅ X\)
and \(ρ_X : X ⊗ \mathbf{1}_{\mathcal{C}} ≅ X\).

These associators and unitors satisfy the pentagon and triangle identities.
\end{definition}

\begin{definition}[Monoidal closed category]
  \label{def:monoidal-closed}
  \lean{CategoryTheory.MonoidalClosed}
  \uses{def:monoidal-category}
  \mathlibok
  
\end{definition}

\begin{definition}[Cartesian closed category] 
  \label{def:cartesian-closed}
  \lean{CategoryTheory.CartesianClosed}
  \uses{def:monoidal-closed, def:chosen-finite-products}
  \mathlibok
  A category \(\mathcal{C}\) is \emph{cartesian closed} if it has finite products and every object is exponentiable.

  We define this as a monoidal closed category with respect to the cartesian monoidal structure.
\end{definition}

\section{Subobject Classifiers}

\begin{definition}[Subobject Classifier]
  \label{def:subobject-classifier}
  \lean{CategoryTheory.ChosenSubobjectClassifier}
  \uses{def:chosen-finite-limits}
  \notready

  In a category $C$ with chosen finite limits, a chosen \emph{subobject classifier} consists of:
  \begin{enumerate}
  \item An object $\Omega$
  \item A monomorphism $\text{true} : 1 \to \Omega$ 
  \item For every monomorphism $m : U \to X$, a unique assigned morphism $\chi_m : X \to \Omega$ (the \emph{characteristic morphism}) such that the square
    \[\begin{tikzcd}
        U \ar[r] \ar[d, "m"'] & 1 \ar[d, "\text{true}"] \\
        X \ar[r, "\chi_m"'] & \Omega
      \end{tikzcd}\]
    is a pullback.
  \end{enumerate}
\end{definition}

\section{Elementary Toposes}

\begin{definition}[Elementary Topos]
\label{def:elementary_topos}
\lean{CategoryTheory.ElementaryTopos}
\uses{def:chosen-finite-limits, def:cartesian-closed-category, def:subobject-classifier}
\notready
An \emph{elementary topos} is a category $\mathcal{C}$ that
\begin{enumerate}
\item has chosen finite limits,
\item is cartesian closed for the inherited chosen finite products,
\item has a subobject classifier.
\end{enumerate}
\end{definition}

\begin{definition}[Natural Numbers Object]
\label{def:nno}
\uses{def:elementary_topos}
\notready
In a topos $\mathcal{C}$, a \emph{natural numbers object} is an object $N$ with morphisms $0 : 1 \to N$ and $s : N \to N$ satisfying the appropriate universal property.
\end{definition}

\begin{definition}[Topos with NNO]
\label{def:topos_with_nno}
\uses{def:elementary_topos, def:nno}
\notready
A topos equipped with a natural numbers object.
\end{definition}

\section{Examples}
\subsection{Finite Sets}
\subsection{The Category of Types}
\subsection{Presheaves}
\subsection{Sheaves}

