% In this file you should put the actual content of the blueprint.
% It will be used both by the web and the print version.
% It should *not* include the \begin{document}
%
% If you want to split the blueprint content into several files then
% the current file can be a simple sequence of \input. Otherwise It
% can start with a \section or \chapter for instance.

\section{Introduction}

An \emph{elementary topos} is a Cartesian closed category with finite limits and a subobject classifier. The goal of this project is to formalise this definition in a suitable way. Namely, in mathlib Cartesian closed categories are defined via \emph{chosen} finite products, which realise finite products as an algebraic operation in a category; for each finite product diagram we \emph{choose} a specific object to serve as the product, rather than only asking for the existence of such a product. Classically, we can still, using the axiom of choice, get chosen limits from existence.

The first part of the project will be extending this to define finite chosen limits in general -- this will allow us to work more naturally in a topos, and keep with the existing definition of Cartesian closedness by getting an instance for chosen finite products from limits.

The second part will be defining a subobject classifier, which we will again see as a chosen subobject classifier -- an object of truth values $\Omega$ in the category with a monomorphism $\operatorname{true} : 1 \rightarrow \Omega$ and a unique characteristic morphism $\chi_{U} : X \rightarrow \Omega$ for any monomorphism $U \rightarrow X$.

If time permits, we will also define a chosen natural numbers object, allowing us to also consider NNO-topoi.

After the definition, the goal will be to apply it and give instances for some common examples, like finite sets, $\operatorname{Type}$, presheaves, and sheaves.

\section{Chosen Finite Limits}

The first step is to generalise chosen finite products to general finite limit cones. Note that chosen limits are also limits in the sense of there existing a limit cone. Being a chosen limit is a strictly stronger property.

\begin{definition}[Chosen Limit]
  \label{def:chosen-limit}
  \lean{CategoryTheory.Limits.ChosenLimit}
  \leanok
Let $F : J \to C$ be a functor. A \emph{chosen limit} for $F$ is a specified limit cone for $F$.
\end{definition}

\begin{proposition}[Chosen Limit implies Limit]
\label{prop:chosen-limit-implies-limit}
\lean{CategoryTheory.Limits.hasLimit_From_chosenLimit} 
\leanok
\uses{def:chosen-limit}
If we have a chosen limit for a functor $F$, then $F$ has a limit.
\end{proposition}

\begin{definition}[Chosen Limits of Shape]
  \label{def:chosen-limits-of-shape}
  \lean{CategoryTheory.Limits.ChosenLimitsOfShape}
  \leanok
A category $C$ has \emph{chosen limits of shape} $J$ if every functor $F : J \to C$ has a chosen limit.
\end{definition} 

\begin{definition}[Chosen Finite Limits]
\label{def:chosen-finite-limits}
\lean{CategoryTheory.Limits.ChosenFiniteLimits}
\leanok
A category $C$ has \emph{chosen finite limits} if it has chosen limits of shape $J$ for every finite category $J$, i.~e.~ for every finite limit cone.
\end{definition}

Note that Mathlib already provides a concept of chosen finite product it uses internally for Cartesian closed categories.

\begin{definition}[Chosen Finite Products]
\label{def:chosen-finite-products}
\lean{CategoryTheory.ChosenFiniteProducts}
\mathlibok
A category $C$ has \emph{chosen finite products} if it is equipped with:
\begin{itemize}
\item A choice of a limit binary fan (product) for any two objects of the category.
\item A choice of a terminal object.
\end{itemize}
More precisely, this consists of chosen limit cones for the pair functor $X \leftarrow \cdot \rightarrow Y$ and for the empty functor from the empty category.
\end{definition}

Note that our definition is compatible, so that a category with chosen finite limits in particular has an instance of chosen finite products.

\begin{lemma}[Chosen Finite Products from Limits]
  \label{lem:chosen-finite-products-from-limits}
\lean{CategoryTheory.Limits.ChosenFiniteProducts_From_ChosenFiniteLimits}
\uses{def:chosen-finite-limits, def:chosen-finite-products}
If a category $C$ has chosen finite limits, then it has chosen finite products.
\end{lemma}

\begin{proof}
  We construct the limit cones for the pair and the terminal object.
  \leanok
\end{proof}

\section{Subobject Classifiers}
\section{Elementary Toposes}
\section{Examples}
\subsection{Finite Sets}
\subsection{The Category of Types}
\subsection{Presheaves}
\subsection{Sheaves}
