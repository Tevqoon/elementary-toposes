% In this file you should put all LaTeX macros and settings to be used both by
% the pdf version and the web version.
% This should be most of your macros.

% The theorem-like environments defined below are those that appear by default
% in the dependency graph. See the README of leanblueprint if you need help to
% customize this. 
% The configuration below use the theorem counter for all those environments
% (this is what the [theorem] arguments mean) and never resets it.
% If you want for instance to number them within chapters then you can add
% [chapter] at the end of the next line.
\newtheorem{theorem}{Theorem}
\newtheorem{proposition}[theorem]{Proposition}
\newtheorem{lemma}[theorem]{Lemma}
\newtheorem{corollary}[theorem]{Corollary}

\theoremstyle{definition}
\newtheorem{definition}[theorem]{Definition}

\providecommand{\N}{\mathbb{N}}
\providecommand{\Z}{\mathbb{Z}}
\providecommand{\Q}{\mathbb{Q}}
\providecommand{\R}{\mathbb{R}}
\providecommand{\C}{\mathbb{C}}
\providecommand{\F}{\mathbb{F}}
\providecommand{\bH}{\mathbb{H}}
\providecommand{\HH}{\mathbb{H}}
\providecommand{\E}{\mathbb{E}}
\providecommand{\A}{\mathbb{A}}
\let\P\relax
\providecommand{\P}{\mathbb{P}}
\providecommand{\lt}{<}
\providecommand{\gt}{>}
\providecommand{\lint}{\int\limits}
\providecommand{\liint}{\iint\limits}
\providecommand{\liiint}{\iiint\limits}
\providecommand{\lidotsint}{\idotsint\limits}
\providecommand{\olint}{\oint\limits}

\DeclareMathOperator{\id}{id}
\DeclareMathOperator{\Sim}{Sim}
\DeclareMathOperator{\rang}{rang}
\DeclareMathOperator{\sgn}{sgn}
\DeclareMathOperator{\im}{im}
\DeclareMathOperator{\Lin}{Lin}
\DeclareMathOperator{\enm}{End}
\DeclareMathOperator{\aut}{Aut}
\DeclareMathOperator{\Int}{Int}
\DeclareMathOperator{\grad}{grad}
\DeclareMathOperator{\divv}{div}
\DeclareMathOperator{\rot}{rot}
\DeclareMathOperator{\Cl}{Cl}
\DeclareMathOperator{\Fr}{Fr}
\DeclareMathOperator{\dom}{dom}
\DeclareMathOperator{\cod}{cod}
\DeclareMathOperator{\argmax}{argmax}
\DeclareMathOperator{\argmin}{argmin}
\DeclareMathOperator{\tr}{tr}
\DeclareMathOperator{\charf}{char}
\DeclareMathOperator{\graf}{Graf}
\DeclareMathOperator{\gal}{Gal}
\DeclareMathOperator{\ob}{Ob}
\DeclareMathOperator{\gend}{End}
\DeclareMathOperator{\meja}{Meja}
\DeclareMathOperator{\Res}{Res}
\let\Im\relax
\DeclareMathOperator{\Im}{Im}
\let\Re\relax
\DeclareMathOperator{\Re}{Re}
\DeclareMathOperator{\Hom}{Hom}
\DeclareMathOperator{\tang}{T}
\DeclareMathOperator{\spec}{Spec}
\DeclareMathOperator*{\colim}{colim}
\DeclareMathOperator{\fl}{fl}
\DeclareMathOperator{\codim}{codim}
\DeclareMathOperator{\supp}{supp}
\DeclareMathOperator{\Max}{Max}
\DeclareMathOperator{\Spec}{Spec}
\DeclareMathOperator{\Mod}{Mod}

\providecommand{\pot}{\mathcal{P}}
\providecommand{\rel}[1]{\mathrel{#1}}
\providecommand{\set}[1]{\left\{#1\right\}}
\providecommand{\setb}[2]{\left\{#1~\middle\vert~#2\right\}}
\providecommand{\abs}[1]{\left\lvert #1\right\rvert}
\providecommand{\norm}[1]{\left\lVert #1\right\rVert}
\providecommand{\floor}[1]{\left\lfloor #1\right\rfloor}
\providecommand{\ceil}[1]{\left\lceil #1\right\rceil}
\providecommand{\eval}[3]{\left. #1\right|_{#2}^{#3}}
\providecommand{\skl}[1]{\left\langle #1\right\rangle}
\providecommand{\kvoc}[2]{{#1/_{#2}}}
\providecommand{\br}[1]{\left(#1\right)}

%Quiver commands
%A TikZ style for curved arrows of a fixed height, due to AndréC.
% \tikzset{curve/.style={settings={#1},to path={(\tikztostart)
%     .. controls ($(\tikztostart)!\pv{pos}!(\tikztotarget)!\pv{height}!270:(\tikztotarget)$)
%     and ($(\tikztostart)!1-\pv{pos}!(\tikztotarget)!\pv{height}!270:(\tikztotarget)$)
%     .. (\tikztotarget)\tikztonodes}},
%     settings/.code={\tikzset{quiver/.cd,#1}
%         \def\pv##1{\pgfkeysvalueof{/tikz/quiver/##1}}},
%     quiver/.cd,pos/.initial=0.35,height/.initial=0}

% % TikZ arrowhead/tail styles.
% \tikzset{tail reversed/.code={\pgfsetarrowsstart{tikzcd to}}}
% \tikzset{2tail/.code={\pgfsetarrowsstart{Implies[reversed]}}}
% \tikzset{2tail reversed/.code={\pgfsetarrowsstart{Implies}}}
% % TikZ arrow styles.
% \tikzset{no body/.style={/tikz/dash pattern=on 0 off 1mm}}

%x-arrows
\makeatletter
\newcommand*{\relrelbarsep}{.386ex}
\newcommand*{\relrelbar}{%
  \mathrel{%
    \mathpalette\@relrelbar\relrelbarsep
  }%
}
\newcommand*{\@relrelbar}[2]{%
  \raise#2\hbox to 0pt{$\m@th#1\relbar$\hss}%
  \lower#2\hbox{$\m@th#1\relbar$}%
}
\providecommand*{\rightrightarrowsfill@}{%
  \arrowfill@\relrelbar\relrelbar\rightrightarrows
}
\providecommand*{\leftleftarrowsfill@}{%
  \arrowfill@\leftleftarrows\relrelbar\relrelbar
}
\providecommand*{\xrightrightarrows}[2][]{%
  \ext@arrow 0359\rightrightarrowsfill@{#1}{#2}%
}
\providecommand*{\xleftleftarrows}[2][]{%
  \ext@arrow 3095\leftleftarrowsfill@{#1}{#2}%
}
\makeatother
